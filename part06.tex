\section{Универсальные семейства хеш-функций}
Построение универсального семейства для
целочисленных ключей и для строк.
Оценка времени поиска в хеш-таблице
при использовании метода цепочек для универсального семейства.
Совершенное хеширование с помощью
универсального семейства хеш-функций.

\subsection{Решение}
Множество функций $\cH$
из множества ключей $\mathcal{K}$
в $\{0, \ldots, m - 1\}$
называется
\emph{универсальным семейством хеш-функций},
если для любой пары различных ключей
$k_1$ и $k_2$
\[ \Pr_{h \in \cH} \brackets{h(k_1) = h(k_2)} \le \frac{1}{m} \]
(хеш-функция случайно выбирается один раз)

\begin{theorem}
    Среднее время безуспешного поиска в хеш-таблице при
    универсальном хешировании составляет $\Theta(1 + \alpha)$.
\end{theorem}
\begin{proof}
    \[
        X(i) =
        \begin{cases}
            1 & h(k_i) = h(k) \\
            0 & h(k_i) \ne h(k) \\
        \end{cases}
    \]

    По определению универсального семейства,
    и поскольку $k$ отсутствует в хеш-таблице
    $\E_h(X(i)) \le 1 / m$.

    Поэтому
    \begin{gather*}
        \E \brackets{\sum_{i=1}^n X(i)}
        = \sum_{i=1}^n \E X(i)
        \le \frac{n}{m} = \alpha
    \end{gather*}

    Хотя бы одно действие мы точно произведём,
    поэтому $\Theta(1 + \alpha)$.
\end{proof}

\begin{theorem}
    Среднее время успешного поиска в хеш-таблице при
    универсальном хешировании составляет $\Theta(1 + \alpha)$.
\end{theorem}
\begin{proof}
    Аналогично безуспешному поиску,
    но однин из $X(i)$ будет строго 1,
    поэтому
    \begin{gather*}
        \E \brackets{\sum_{i=1}^n X(i)}
        = \sum_{i=1}^n \E X(i)
        \le 1 + \frac{n - 1}{m} \in \Theta(1 + \alpha)
    \end{gather*}
\end{proof}

\subsection{Пример}
\begin{theorem}
    Путь $K = \{0, \ldots, n\}$.
    Выберем простое $p > n$.
    Тогда семейство $\cH$,
    состоящее из функций
    \[ h_{a, b}(k) = \Bigl( (ak + b) \Mod p \Bigr) \Mod m \]
    для $a = 1, \ldots, p - 1$ и $b = 0, \ldots, p - 1$
    будет универсальным.
\end{theorem}
\begin{proof}
    Рассмотрим $k_1 \ne k_2$.
    \begin{align*}
        & t_1 = (ak_1 + b) \Mod p &
        & t_2 = (ak_2 + b) \Mod p \\
    \end{align*}

    Т.к. $k_1 \ne k_2$, то $t_1 \ne t_2$.
    Если предположить, что $t_1 = t_2$,
    то
    \begin{gather*}
        ak_1 + b \equiv ak_2 + b \pmod{p} \\
        a(k_1 - k_2) \equiv 0 \pmod{p} \\
        a \ne 0 \Rightarrow k_1 = k_2 \\
    \end{gather*}

    По $t_1, t_2, k_1, k_2$ можно однозначно восстановить $a, b$:
    \begin{gather*}
        t_1 - t_2 \equiv a(k_1 - k_2) \pmod{p} \\
        a \equiv (t_1 - t_2) \cdot (k_1 - k_2)^{-1} \pmod{p} \\
        b \equiv t_1 - a k_1 \pmod{p} \\
    \end{gather*}
    (кольцо по модулю $p$ --- поле, в нём есть деление).

    При всех возможных парах $a, b$
    каждая пара $t_1, t_2$ встречается ровно один раз.
    Тогда если выбирать $h_{a, b}$ случайно и равномерно из $\cH$,
    то пары $t_1, t_2$ будут случайно и равномерно
    распределены на $\{0, \ldots, p - 1\}$,
    при этом $t_1 \ne t_2$.

    Тогда вероятность того, что хеш-коды ключей совпадут,
    равна вероятности того, что два различных числа
    из $\{0, \ldots, p - 1\}$ окажутся равны по модулю $m$.
    Для фиксированного $t_1$ количество таких $t_2$ не превосходит
    \[
        \ceil{\frac{p}{m}} - 1
        \le \frac{p + m - 1}{m} - 1
        = \frac{p - 1}{m}
    \]

    Тогда вероятность получить коллизию равна
    \[
        \frac{1}{p} \cdot \frac{p - 1}{m} \le \frac{1}{m}
    \]
\end{proof}

\subsection{Совершенное хеширование}
