\section{Поиск подстроки в строке}
Наивный алгоритм.
Алгоритм Рабина-Карпа.
$z$-функция, префикс-функция,
алгоритм Кнута-Морриса-Пратта.

\subsection{Конспект}
Наивный алгоритм --- очевидно.

Алгоритм Рабина-Карпа: полиномиальное хеширование.
$\O(n + m)$.

$z$-функция:
$z_i$ --- максимальная длина префикса строки,
который совпадает с префиксом её $i$-го суффикса.
префикс- или $\pi$-функция:
$\pi_i$ --- максимальная длина префикса $(i - 1)$-го префикса строки,
который совпадает с суффиксом её $i$-го префикса.
$z$- и префикс-функцию можно посчитать за $\O(n)$:

\bigskip
\noindent
\begin{minipage}{\textwidth}
\begin{minted}{rust}
pub fn calc_zfun_inplace(s: &[impl Eq], z: &mut [usize]) {
    let n = s.len();
    assert_eq!(n, z.len());
    if n == 0 {
        return;
    }
    z[0] = n;
    if n == 1 {
        return;
    }
    // Поддерживаем самый правый совпадающий с префиксом диапазон,
    // может быть нулевой длины
    let mut l = 1;
    let mut r = 1;
    z[1] = r - l;
    for i in 1..n {
        let mut k = 0;
        if i < r {
            // s[i..r] = s[i - l..r - l]
            k = z[i - l].min(r - i);
        }
        while i + k < n && s[k] == s[i + k] {
            k += 1;
        }
        z[i] = k;
        if i + k > r {
            l = i;
            r = i + k;
        }
    }
}
\end{minted}
\end{minipage}

Пояснение:
\begin{center}
    \begin{tikzpicture}

    \end{tikzpicture}
\end{center}

\bigskip
\noindent
\begin{minipage}{\textwidth}
\begin{minted}{rust}
pub fn calc_pfun_inplace(s: &[impl Eq], p: &mut [usize]) {
    let n = s.len();
    assert_eq!(n, p.len());
    if n == 0 {
        return;
    }
    p[0] = 0;
    for i in 1..n {
        let mut k = p[i - 1];
        while k != 0 && s[i] != s[k] {
            k = p[k - 1];
        }
        if s[i] == s[k] {
            k += 1;
        }
        p[i] = k;
    }
}
\end{minted}
\end{minipage}

Кнута-Морриса-Пратта:
для поиска паттерна в строке
запишем паттерн как префикс строки
через специальный разделитель
и будем искать по префикс-функции.
$\O(n + m)$, очевидно.
