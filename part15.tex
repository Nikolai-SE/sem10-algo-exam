\section{Суффиксные структуры}
Суффиксное дерево: определение, поиск подстроки.
Суффиксный массив: определение, поиск подстроки,
построение за $\O(n \log n)$.

\subsection{Суффиксное дерево}
Суффиксное дерево --- бор по всем суффиксам строки.
Количество вершин --- $\O(n^2)$, за столько же строится.
Сжатое суффиксное дерево --- рёбра содержат не символы,
а строки (точнее, срезы исходной строки),
сжимаем рёбра, которые не делятся.

\begin{theorem}
    Размер сжатого суффиксного дерева
    --- $\O(n)$.
\end{theorem}
\begin{proof}
    Будем строить дерево.
    Если при взятии очередного суффикса
    мы добавляем ветвление, то это добавленное ветвление
    --- единственное на новом пути.
    Тогда всего ветвлений добавлено не более $n - 1$,
    следовательно, рёбер всего не более $2n - 1$.
\end{proof}

Существует алгоритм построения за $\O(n)$.

Поиск подстроки --- идём по суффиксному дереву,
все суффиксы, которые входят в поддерево вершины подстроки
--- вхождения.

Можно искать наибольшую общую подстроку
через разные маркеры.

\subsection{Суффиксный массив}
Массив, в котором все суффиксы отсортированы лексикографически.
Суффиксом однозначно задаётся его смещением,
поэтому память --- $\O(n)$.

Построение за $\O(n \log n)$: сортировка подсчётом
сначала по первым двум символам,
потом по текущему положению и суффиксу суффикса
(т.к. они все --- суффиксы друг друга).

Массив LCP (Longest Common Prefix)
--- LCP соседних суффиксов в массиве.
LCP произвольных суффиксов это RMQ между ними.
Существует алгоритм вычисления LCP за $\O(n)$
