\section{Методы решения NP-полных задач}
Методы решения NP-полных задач.
Backtracking, Branch \& Bound, параметризованные алгоритмы.
Приближенные алгоритмы для NP-полных задач.
2-приближенный алгоритм для вершинного покрытия.
2-приближенный алгоритм для метрического коммивояжёра.
Алгоритм Кристофидеса-Сердюкова.
Константное приближение TSP влечёт P = NP.
$\log(n)$-приближенный алгоритм для покрытия множествами.
Формулировка жадной гипотезы о надстроке.

\subsection{Конспект}
TSP --- задача коммивояжёра:
сщуествует ли простой цикл,
проходящий через все вершины.

\bigskip

2-приближённый вершинного покрытия
--- смотрим на каждое ребро, если оно уже покрывается,
то всё ок, если нет, то берём обе его вершины.

\bigskip

Метрическая TSP: на входе полный граф,
веса между вершинами --- расстояния (математические).
Тогда 2-приближением будет построить MST,
провести обход по нему в порядке DFS,
и удалить повторения вершин из пути.
Истинный кратчайший путь $S$
как минимум содержит в себе какое-то
остовное дерево, следовательно,
не меньше MST.
А путь, который мы нашли, это удвоенное MST.

\bigskip

Алгоритм Кристофидеса-Сердюкова:
в $T$ количество вершин нечётной степени чётно.
Найдём совершенное паросочетание
минимального веса (не умеем это делать) $M$
на этих вершинах
и добавим его рёбра в $T$,
получился мультиграф с вершинами чётной степени.
Теперь существует эйлеров цикл,
возьмём его и удалим повторы.

$2 |M| \le |S|$:
построим два паросочетания $M_1$ и $M_2$ на вершинах $M$,
идя по $S$ и чередуя рёбра (т.е. $M_1 \cup M_2 \subset S$).
Тогда $|M_1| + |M_2| \le |S|$,
а $|M| \le \min(|M_1|, |M_2|)$.
Следовательно, это будет $3/2$-приближением.
