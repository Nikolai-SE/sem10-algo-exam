\section{Структура бор}
Бор. Алгоритм Ахо-Корасик.

\subsection{Конспект}
Бор --- дерево, где на рёбрах символы,
а пути от корня до вершин --- строки.
Вершины помечены тем, хранится ли такая строка
в множестве.

Алгоритм Ахо-Корасик:
нужно найти все вхождения строк $s_i$
из словаря в строку $t$.
Для этого построим бор
и добавим к нему суффиксные ссылки.
Суффиксная ссылка --- ребро из вершины бора в такую вершину,
строка которой соответствует максимально возможному суффиксу
исходной вершины.
Такая суффиксная ссылка у каждой вершины может быть только одна.
Используем суффиксные ссылки как переходы в случае неудачи в КА,
построенном на боре, при проходе по строке.

Как искать строки:
\begin{enumerate}
    \item Строка в терминальной вершине --- нашли
    \item Для всех строк, которые могут встретиться как подстроки,
    делаем сжатые суффиксные ссылки,
    т.е. ссылки не на максимальный суффикс,
    а на максимальный, являющийся строкой словаря.
    Их можно найти за $\O(|T|)$ ($T$ --- наш бор, trie).
\end{enumerate}

Как построить суффиксные ссылки за $\O(|T| \cdot |\Sigma|)$:
обходим бор по уровням (в порядке BFS),
следующий уровень либо продолжает суффиксную ссылку предыдущего,
либо поднимает свою выше, как с префикс-функцией.
Поскольку это бор, то одной строке соответствует одна вершина.

Время работы всего алгоритма Ахо-Корасик:
$\O(\sum |s_i| + |t| + \#\text{ответов})$
