\section{Числовые алгоритмы}
Cложение, умножение, деление длинных чисел.
Модульная арифметика: сложение, умножение,
возведение в степень, алгоритм Евклида,
расширенный алгоритм Евклида, деление.

\subsection{Алгоритмы арифметики}
Числа представляются как двоичные строки.
$n$ --- длина числа в битах.

Сложение --- очевидный алгоритм за $\O(n)$.

В процессорах умножение обычно
реализуется следующим алгоритмом в столбик
за $\O(n^2)$:
\begin{minted}{rust}
fn mul(mut a: Int, mut b: Int) -> Int {
    let mut r: Int = 0;
    for _ in 0..Int::BITS {
        if b & 1 == 1 {
            r += a;
        }
        a <<= 1;
        b >>= 1;
    }
    r
}
\end{minted}

Деление и получение остатка реализуется
алгоритмом в столбик
за $\O(n^2)$
(точнее, за $\O(n \cdot (n - m))$,
где $m$ --- старший бит $b$, считая с 0, $m < n$):
\begin{minted}{rust}
fn div_rem(mut a: Int, mut b: Int) -> (Int, Int) {
    let mut r: Int = 0;
    let mut maxbit = 0;
    for i in 0..Int::BITS {
        if b & (1 << i) != 0 {
            maxbit = i;
        }
    }
    for i in (0..Int::BITS - maxbit).rev() {
        let b = b << i;
        if a >= b {
            r |= 1 << i;
            a -= b;
        }
    }
    (r, a)
}
\end{minted}

Есть алгоритм Карацубы для умножения
за $\O(n^{\log_2 3})$.
Он основан на следующем:
\begin{gather*}
    (ax + b) \cdot (cx + d)
    = ac x^2 + (ad + bc) \cdot x + bd \\
    (a + b) \cdot (c + d)
    = ac + ad + bc + bd \\
    ad + bc = (a + b) \cdot (c + d) - ac - bd
\end{gather*}

Поэтому при дроблении задачи в 2 раза
достаточно провести 3 умножения частей
и $\O(1)$ сложений, а не 4 умножения,
т.е.
\begin{gather*}
    T(n) = 3 \cdot T(n/2) + \O(n) \\
    T(n) \in \O(n^{\log_2 3})
\end{gather*}

Существует алгоритм умножения,
основанный на быстром преобразовании Фурье,
за $\O(n \log n)$.

Базовый алгоритм Евклида:
\begin{minted}{rust}
fn gcd(a: Int, b: Int) -> Int {
    if a < b {
        gcd(b, a)
    } else if b == 0 {
        a
    } else {
        gcd(b, a - b)
    }
}
\end{minted}

Обычно пишут как
\begin{minted}{rust}
fn gcd(a: Int, b: Int) -> Int {
    if b == 0 {
        a
    } else {
        gcd(b, a % b)
    }
}
\end{minted}

\begin{theorem}
    Алгоритм Евклида находит НОД.
\end{theorem}
\begin{proof}
    Пусть $a = bq + r$.
    Пусть $a \equiv 0 \pmod c$
    и $b \equiv 0 \pmod c$.
    Тогда $r \equiv 0 \pmod c$,
    т.к. $r = a - bq$.
    Тогда множества общих делителей
    у $(a, b)$ и у $(b, r)$ совпадают.
\end{proof}

\begin{theorem}[Теорема Безу]
    Всегда существуют целые $x$ и $y$
    такие, что $ax + by = \gcd(a, b)$.
\end{theorem}
\begin{proof}
    Расширенный алгоритм Евклида.
    \begin{minted}{rust}
    fn gcd_ext(a: Int, b: Int) -> (Int, Int, Int) {
        if a < b {
            let (g, x, y) = gcd_ext(b, a);
            (g, y, x)
        } else if b == 0 {
            (a, 1, 0)
        } else {
            let (g, x, y) = gcd_ext(b, a - b);
            (g, y, x - y)
        }
    }
    \end{minted}
\end{proof}

Расширенный алгоритм Евклида можно записать через модуль:
\begin{minted}{rust}
fn gcd_ext_mod(a: Int, b: Int) -> (Int, Int, Int) {
    if b == 0 {
        (a, 1, 0)
    } else {
        let q = a / b;
        let (g, x, y) = gcd_ext_mod(b, a % b);
        (g, y, x - q * y)
    }
}
\end{minted}

\begin{theorem}
    $|x| \le b$ и $|y| \le a$
\end{theorem}
\begin{proof}
    Если $a \equiv 0 \pmod b$,
    то рекурсивный вызов вернёт $(b, 1, 0)$.

    По индукции предположим, что для рекурсивного вызова выполняется.
    Тогда
    \begin{gather*}
        q = \floor{\frac{a}{b}} \\
        r = a \Mod b \\
        \gcd(a, b) = x' \cdot b + y' \cdot r \\
        |x'| \le r \\
        |y'| \le b \\
        x = y' \\
        y = x' - qy' \\
        |x| = |y'| \le b \\
        |y| \le |x'| + |qy'| \le r + qb = a
    \end{gather*}
\end{proof}

Возведение в степень --- алгоритм за $\O(M \log p)$,
где $M$ --- сложность умножения, а $p$ --- степень.

\subsection{Модульная арифметика}
Сложение, умножение, степень --- очевидно.

Деление --- через расширенный алгоритм Евклида
и поиск обратного:
\begin{gather*}
    a / b = a \cdot b^{-1} \\
    b \cdot b^{-1} \equiv 1 \pmod m \\
\end{gather*}

Для существования такого $q$ нужно,
чтобы $\gcd(b, m) = 1$, т.к. иначе
$1 \ne \gcd(b, m) \mid bq$.

Воспользуемся расширенным алгоритмом Евклида и найдём $x$ и $y$:
\[ bx + my \equiv bx \equiv 1 \pmod m \]
это $x$ и будет искомым $q$.
Нужно модифицировать $\gcd$ для использования модульного
сложения и умножения:
\begin{minted}{rust}
fn gcd_ext_mod(a: Int, b: Int, m: Int) -> (Int, Int, Int) {
    if a < b {
        let (g, x, y) = gcd_ext_mod(b, a, m);
        (g, y, x)
    } else if b == 0 {
        (a, 1, 0)
    } else {
        let (g, x, y) = gcd_ext_mod(b, (a + m - b) % m, m);
        (g, y, (x + m - y) % m)
    }
}
\end{minted}

Если $m$ --- простое, то обратный элемент есть всегда.
