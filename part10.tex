\section{Задача линейного программирования и симплекс-метод}
Линейное программирование.
Общий вид задачи, матричная форма и
сведение между различными представлениями.
Вершины: равносильность двух определений и достижение максимума.
Симплекс-метод, нахождение начальной точки.
Двойственность: построение двойственной
задачи, теорема о двойственности --- слабая (с доказательством)
и сильная (без доказательства) формулировки.

\subsection{Задача}
Есть набор переменных, нужно присвоить им вещественные значения
с учётом линейных ограничений
и максимизируя / минимизируя линейную функцию.

\begin{align*}
    &
    \left\{
    \begin{aligned}
        & \sum_{i=1}^n c_i x_i \to \max \\
        & \forall j \in \{1, \ldots, m\}.~\sum_{i=1}^n a_{ji} x_i \le b_j \\
        & \forall i \in \{1, \ldots, n\}.~x_i \ge 0 \\
    \end{aligned}
    \right.
    &&
    \left\{
    \begin{aligned}
        C^T \cdot X & \to \max \\
        A \cdot X & \le B \\
        X & \ge 0 \\
    \end{aligned}
    \right.
\end{align*}

Пример --- максимизация прибыли,
если на складе есть материалы,
из которых производятся товары
заранее известной стоимости.

Стоит обратить внимание, что неравенства нестрогие.

\subsection{Разные виды}
\begin{itemize}
    \item Целевая функция максимизируется или минимизируется
    (эквивалентно, т.к. просто $c_i$ разного знака);

    \item Ограничения неравенством или равенством:
    для $\sum_{i=1}^n a_i x_i \le b$
    вводим $s$ и пишем
    \[
        \left\{
        \begin{aligned}
            & \sum_{i=1}^n a_i x_i + s = b \\
            & s \ge 0 \\
        \end{aligned}
        \right.
    \]

    Обратно --- для $A = B$ пишем $A \le B$ и $A \ge B$ ($-A \le -B$);

    \item Неограниченная переменная сводится к двум неотрицательным:
    $x = x^+ - x^-$, $x^+ \ge 0$, $x^- \ge 0$.
\end{itemize}

\paragraph{Стандартный вид:}
целевая функция минимизируется,
все переменные неотрицательные,
ограничения --- уравнения.

% Линейные ограничения задают гиперплоскости,
% которые содержат грани полиэдра (многогранника),
% в котором лежат все точки, попадающие под ограничения.
% Ограничения могут быть не замкнуты,
% тогда функция может быть бесконечной,
% либо точек, попадающих под ограничения,
% может не быть.
% Если же ограничения замкнуты,
% тогда ответ лежит в одной из вершин.

% Симплекс-метод:
% найти некоторую вершину,
% дальше идти в сторону тех соседей,
% для которых целевая функция увеличивается.

\subsection{Двойственная задача}
\paragraph{Пример}
\[
    \left\{
    \begin{aligned}
        x_1 + 6 x_2 & \to \max \\
        x_1 & \le 200 \\
        x_2 & \le 300 \\
        x_1 + x_2 & \le 400 \\
        x_1, x_2 & \ge 0 \\
    \end{aligned}
    \right.
\]

Решение этой задачи:
$x_1 = 100, x_2 = 300$,
тогда $x_1 + 6 x_2 = 1900$.
Можно доказать, что это решение, сложив
второе и третье ограничения с коэффициентами 5 и 1:
\begin{gather*}
    5 x_2 + x_1 + x_2 \le 5 \cdot 300 + 400 \\
    x_1 + 6 x_2 \le 1900
\end{gather*}

Хотим найти такие коэффициенты для оценки в общем случае.
Обозначим коэффициенты при неравенствах
как $y_1, y_2, y_3$.
Хотим минимизировать
$y_1 \cdot x_1 + y_2 \cdot x_2 + y_3 \cdot (x_1 + x_2)$,
при этом $y_1, y_2, y_3 \ge 0$
(иначе поменяется знак неравенства).
Тогда получаем неравенство
\[
    (y_1 + y_3) x_1 + (y_2 + y_3) x_2 \le 200 y_1 + 300 y_2 + 400 y_3
\]
При этом хотим, чтобы левая часть совпала с целевой функцией,
т.е.
\[
    x_1 + 6 x_2 \le 200 y_1 + 300 y_2 + 400 y_3
\]
Для этого требуются ограничения
\[
    \left\{
    \begin{aligned}
        y_1, y_2, y_3 & \ge 0 \\
        y_1 + y_3 & \ge 1~\text{(неравенство, т.к. $y$ только справа)} \\
        y_2 + y_3 & \ge 6
    \end{aligned}
    \right.
\]

Чтобы оценка была как можно более точной,
нужно минимизировать $200 y_1 + 300 y_2 + 400 y_3$.
Снова получили задачу линейного программирования:
\[
    \left\{
    \begin{aligned}
        200 y_1 + 300 y_2 + 400 y_3 & \to \min \\
        y_1 + y_3 & \ge 1 \\
        y_2 + y_3 & \ge 6 \\
        y_1, y_2, y_3 & \ge 0 \\
    \end{aligned}
    \right.
\]

\paragraph{Общий случай}
Всякое решение двойственной задачи
--- оценка для прямой задачи,
поэтому пара решений с совпадающими
значениями будет оптимальной.

Если прямая задача:
\[
    \left\{
    \begin{aligned}
        & \sum_{i=1}^n c_i x_i \to \max \\
        & \forall j \in \{1, \ldots, m\}.~\sum_{i=1}^n a_{ji} x_i \le b_j \\
        & \forall i \in \{1, \ldots, n\}.~x_i \ge 0 \\
    \end{aligned}
    \right.
\]

То двойственная --- это оценка для прямой, и будет иметь вид:
\[
    \left\{
    \begin{aligned}
        & \sum_{j=1}^m b_i y_i \to \min \\
        & \forall i \in \{1, \ldots, n\}.~\sum_{j=1}^m a_{ji} y_j \ge c_i \\
        & \forall j \in \{1, \ldots, m\}.~y_j \ge 0 \\
    \end{aligned}
    \right.
\]

Матричный вид:
\begin{align*}
    &
    \left\{
        \begin{aligned}
            C^T \cdot X & \to \max \\
            A \cdot X & \le B \\
            X & \ge 0 \\
        \end{aligned}
    \right.
    &&
    \left\{
        \begin{aligned}
            B^T \cdot Y & \to \min \\
            A^T \cdot Y & \ge C \\
            Y & \ge 0
        \end{aligned}
    \right.
\end{align*}

Из матричного вида очевидно,
что двойственная задача к двойственной задаче
--- прямая задача.
