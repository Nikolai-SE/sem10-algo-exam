\section{Задача линейного программирования и симплекс-метод}
Линейное программирование.
Общий вид задачи, матричная форма и
сведение между различными представлениями.
Вершины: равносильность двух определений и достижение максимума.
Симплекс-метод, нахождение начальной точки.
Двойственность: построение двойственной
задачи, теорема о двойственности --- слабая (с доказательством)
и сильная (без доказательства) формулировки.

\subsection{Задача}
Есть набор переменных, нужно присвоить им вещественные значения
с учётом линейных ограничений
и максимизируя / минимизируя линейную функцию.

\begin{align*}
    &
    \left\{
    \begin{aligned}
        & \sum_{i=1}^n c_i x_i = \max \\
        & \forall j \in \{1, \ldots, m\}.~\sum_{i=1}^n a_{ji} x_i \le b_j \\
        & \forall i \in \{1, \ldots, n\}.~x_i \ge 0 \\
    \end{aligned}
    \right.
    &&
    \left\{
    \begin{aligned}
        C^T \cdot X & = \max \\
        A \cdot X & \le B \\
        X & \ge 0 \\
    \end{aligned}
    \right.
\end{align*}

Пример --- максимизация прибыли,
если на складе есть материалы,
из которых производятся товары
заранее известной стоимости.

Стоит обратить внимание, что неравенства нестрогие.

\subsection{Разные виды}
\begin{itemize}
    \item Целевая функция максимизируется или минимизируется
    (эквивалентно, т.к. просто $c_i$ разного знака);

    \item Ограничения неравенством или равенством:
    для $\sum_{i=1}^n a_i x_i \le b$
    вводим $s$ и пишем
    \[
        \left\{
        \begin{aligned}
            & \sum_{i=1}^n a_i x_i + s = b \\
            & s \ge 0 \\
        \end{aligned}
        \right.
    \]

    Обратно --- для $A = B$ пишем $A \le B$ и $A \ge B$ ($-A \le -B$);

    \item Неограниченная переменная сводится к двум неотрицательным:
    $x = x^+ - x^-$, $x^+ \ge 0$, $x^- \ge 0$.
\end{itemize}

\paragraph{Стандартный вид:}
целевая функция минимизируется,
все переменные неотрицательные,
ограничения --- уравнения.

Линейные ограничения задают гиперплоскости,
которые содержат грани полиэдра (многогранника),
в котором лежат все точки, попадающие под ограничения.
Ограничения могут быть не замкнуты,
тогда функция может быть бесконечной,
либо точек, попадающих под ограничения,
может не быть.
Если же ограничения замкнуты,
тогда ответ лежит в одной из вершин.

Симплекс-метод:
найти некоторую вершину,
дальше идти в сторону тех соседей,
для которых целевая функция увеличивается.
