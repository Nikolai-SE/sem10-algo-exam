\section{Простые числа}
Проверка чисел на простоту.
Числа Кармайкла, малая теорема Ферма.
Генерация случайных простых чисел.
Криптография: схемы с закрытым ключом, RSA.
Необходимые факты теории чисел.

\subsection{Тесты}
\begin{theorem}[Малая теорема Ферма]
    Если $p$ --- простое, и $1 \leq a < p$, то
    \[ a^{p - 1} \equiv 1 \pmod p \]
\end{theorem}
\begin{proof}
    Рассмотрим множество ненулевых остатков от деления на $p$:
    $\{1, \ldots, p - 1\}$.
    При умножении любого из них на $a$ мы получим
    остаток из этого же множества.
    При этом никакие два умножения не дадут один результат,
    т.к. если
    $an \equiv am \pmod p$,
    то $a(n - m) \equiv 0 \pmod p$,
    следовательно, $n = m$.

    Тогда
    \[
        \{1, \ldots, p - 1\}
        = \{ a \cdot 1 \Mod p, \ldots, a \cdot (p - 1) \Mod p \}
    \]

    Перемножим все элементы этого множества:
    \begin{gather*}
        (p - 1)! \equiv a^{p - 1} \cdot (p - 1)! \pmod p \\
        \gcd((p - 1)!, p) = 1 \\
        (p - 1)! \cdot (a^{p - 1} - 1) \equiv 0 \pmod p \\
        a^{p - 1} \equiv 1 \pmod p
    \end{gather*}
\end{proof}

Быстрая проверка на простоту --- тест Ферма:
проверить, подходит ли число под малую лемму Ферма.

\begin{theorem}
    Если $\exists a \mid a^{m - 1} \not \equiv 1 \pmod m \land \gcd(a, m) = 1$,
    то чисел, не проходящих тест, хотя бы половина.
\end{theorem}
\begin{proof}
    Возьмём любое $b \mid b^{m - 1} \equiv 1 \pmod m$.
    Рассмотрим $ab$:
    \[
        (ab)^{m - 1}
        \equiv a^{m - 1} \cdot b^{m - 1}
        \equiv a^{m - 1} \not \equiv 1 \pmod m
    \]

    При этом если $c \ne b, c^{m - 1} \equiv 1 \pmod m$,
    то $ab \ne ac$, т.к. $a$ обратимо.
\end{proof}

Поэтому для случайно взятого $1 < a < m$
шанс ошибки в тесте Ферма не превышает $1/2$,
если число вообще может быть отсеяно этим тестом.

\begin{definition}
    Числа Кармайкла --- составные числа,
    которые всегда проходят тест Ферма.
\end{definition}
Чисел Кармайкла мало: $561 = 3 \cdot 11 \cdot 17$
проходит тест Ферма для всех $a$, взаимно простых с собой.

Тест Рабина-Миллера:
пусть $N$ --- нечётное.
Представим $N - 1 = 2^t \cdot u$,
где $u$ --- нечётное.
Выберем случайное $a$,
вычислим $a^{N - 1} \Mod N$.
Найдём первую единицу в последовательности
\[
    a^u \Mod N,
    a^{2u} \Mod N,
    a^{4u} \Mod N, \ldots,
    a^{N - 1} \Mod N
\]
(а если прошли тест Ферма, то она есть),
посмотрим на предыдущее число.
Если оно не является минус единицей,
то это нетривиальный корень из 1,
и, следовательно, $N$ составное.

Тест Рабина-Миллера с вероятностью $3/4$ обнаруживает
любое составное число, в т.ч. числа Кармайкла.

\subsection{Генерация простых чисел}
