\section{Целочисленное линейное программирование}
ILP: пример для максимального паросочетания
и вершинного покрытия в двудольном графе.
Тотальная унимодулярность:
определение и почему это гарантирует целочисленность решения.
Доказательство теоремы Кёнига через двойственность.

\subsection{Двудольный граф}
Максимальное паросочетание:
каждому ребру $e_j$ назначим $x_j$
--- берём или нет это ребро,
и задача выглядит так:
\begin{eqnsystem}
    & \sum_{j=1}^{|E|} x_j \to \max \\
    & \forall i \le |V|.~\sum_{j=1}^{|E|} a_{ij} x_j \le 1 \\
    % & \forall j \le |E|.~x_j \le 1 \\
    & \forall j \le |E|.~x_j \ge 0 \\
\end{eqnsystem}
где $a_{ij}$ --- инцидентность вершины $v_i$ ребру $e_j$.

Минимальное вершинное покрытие:
каждой вершине $v_i$ назначим $y_i$
--- берём или нет эту вершину.
Задача выглядит так:
\begin{eqnsystem}
    & \sum_{i=1}^{|V|} y_i \to \min \\
    & \forall j \le |E|.~\sum_{i=1}^{|V|} a_{ij} y_i \ge 1 \\
    % & \forall i \le |V|.~y_i \le 1 \\
    & \forall i \le |V|.~y_i \ge 0 \\
\end{eqnsystem}
где $a_{ij}$ --- инцидентность вершины $v_i$ ребру $e_j$.

\begin{theorem}[теорема Кёнига]
    Максимальное паросочетание не больше минимального вершинного покрытия
\end{theorem}
\begin{proof}
    См. задачи линейного программирования выше,
    можно заметить, что если
    $A$ --- матрица инцидентности
    (т.е. $a_{ij}$ --- инцидентность $v_i$ и $e_j$),
    то задачи выглядят как
    \begin{align*}
        &
        \begin{system}
            & 1_{|E|}^T \cdot X \to \max \\
            & A \cdot X \le 1_{|V|} \\
            & X \ge 0 \\
        \end{system}
        &&
        \begin{system}
            & 1_{|V|}^T \cdot Y \to \min \\
            & A^T \cdot Y \ge 1_{|E|} \\
            & Y \ge 0 \\
        \end{system}
    \end{align*}
    Т.е. задачи двойственны друг другу,
    следовательно,
    оптимум минимального вершинного покрытия
    равен оптимуму максимального паросочетания.
\end{proof}

\subsection{Унимодулярность}
Тотальная унимодулярность:
определитель каждой квадратной подматрицы
(в т.ч. с пробелами) --- $\pm 1$ или $0$.
\begin{theorem}
    Все вершины многогранника с тотально унимодулярной
    матрицей целочисленны.
\end{theorem}
\begin{proof}
    Рассмотрим вершину $(x_1, \ldots, x_n)$.
    Её система уравнений:
    \begin{align*}
        &
        \begin{system}
            a_{11} x_1 + \ldots + a_{1n} x_n & = b_1 \\
            & \vdots \\
            a_{n1} x_1 + \ldots + a_{nn} x_n & = b_1 \\
        \end{system}
        &&
        A X = B
    \end{align*}

    Эту систему можно решить методом Крамера:
    \[
        x_i = \frac{1}{\det A} \cdot \det
        \begin{bmatrix}
            a_{11} & \cdots & a_{1,i - 1} & b_1 & a_{1,i + 1} & \cdots & a_{1n} \\
            \vdots & \vdots & \vdots & \vdots & \vdots & \vdots & \vdots \\
            a_{n1} & \cdots & a_{n,i - 1} & b_n & a_{n,i + 1} & \cdots & a_{1n} \\
        \end{bmatrix}
    \]

    Поскольку $b_j$ --- целые, и каждый определитель
    квадратной подматрицы это либо $0$, либо $\pm 1$,
    то определитель правой матрицы целый,
    а определитель матрицы в знаменателе --- $\pm 1$,
    поэтому $x_i \in \Z$.
\end{proof}
