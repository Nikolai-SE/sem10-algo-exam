\section{Целочисленное линейное программирование}
ILP: пример для максимального паросочетания
и вершинного покрытия в двудольном графе.
Тотальная унимодулярность:
определение и почему это гарантирует целочисленность решения.
Доказательство теоремы Кёнига через двойственность.

\subsection{Двудольный граф}
Максимальное паросочетание:
каждому ребру $e_j$ назначим $x_j$
--- берём или нет это ребро,
и задача выглядит так:
\[
    \left\{
        \begin{aligned}
            & \sum_{j=1}^{|E|} x_j \to \max \\
            & \forall i \le |V|.~\sum_{j=1}^{|E|} a_{ij} x_j \le 1 \\
            & \forall j \le |E|.~x_j \le 1 \\
            & \forall j \le |E|.~x_j \ge 0 \\
        \end{aligned}
    \right.
\]
где $a_{ij}$ --- инцидентность вершины $v_i$ ребру $e_j$.

Минимальное вершинное покрытие:
каждой вершине $v_i$ назначим $y_i$
--- берём или нет эту вершину.
Задача выглядит так:
\[
    \left\{
        \begin{aligned}
            & \sum_{i=1}^{|V|} y_i \to \min \\
            & \forall j \le |E|.~\sum_{i=1}^{|V|} a_{ij} y_i \ge 1 \\
            & \forall i \le |V|.~y_i \le 1 \\
            & \forall i \le |V|.~y_i \ge 0 \\
        \end{aligned}
    \right.
\]
где $a_{ij}$ --- инцидентность вершины $v_i$ ребру $e_j$.

% TODO
